\documentclass{article}

% ready for submission
\usepackage{neurips_2023}

\usepackage[utf8]{inputenc} % allow utf-8 input
\usepackage[T1]{fontenc}    % use 8-bit T1 fonts
\usepackage{hyperref}       % hyperlinks
\usepackage{url}            % simple URL typesetting
\usepackage{booktabs}       % professional-quality tables
\usepackage{amsfonts}       % blackboard math symbols
\usepackage{nicefrac}       % compact symbols for 1/2, etc.
\usepackage{microtype}      % microtypography
\usepackage{xcolor}         % colors
\usepackage{listings}
\usepackage{color}
\usepackage{graphicx}

\author{
  Sid Patllollu\\
  UCLA Computer Science\\
  \texttt{sidpat@ucla.edu} \And
  Agustín Costa\\
  UCLA Computer Science\\
  \texttt{agustincosta@g.ucla.edu} \And
  Raj Kumar\\
  UCLA Computer Science\\
  \texttt{raj.a.kumar@ucla.edu} \And
  Alexandre Alves\\
  UCLA Computer Science\\
  \texttt{alexcastroalves@icloud.com}
}

\title{Adaptive Dataset Pruning: Dynamic EL2N-based Training for Noisy Datasets}

\begin{document}

\maketitle

\begin{abstract}
Training deep learning models on noisy datasets presents significant challenges for model performance and training efficiency. While existing approaches like EL2N scoring help identify and remove noisy samples, they typically employ static pruning strategies that may discard valuable training data. We propose a dynamic EL2N-based pruning approach that adaptively adjusts pruning strategies throughout the training process. Our method begins by removing the noisiest samples early in training and progressively shifts focus to harder, more informative examples. Experimental results on CIFAR-10 with varying noise levels (10-30\%) demonstrate that our approach maintains comparable accuracy to full dataset training while pruning an average of 24.4\% of data points, outperforming static EL2N pruning methods by up to 12.38\% in accuracy for highly (at 20\%) noisy datasets.
\end{abstract}

\section{Introduction}
Deep learning models have shown remarkable success across various domains, but their performance heavily depends on the quality of training data. In real-world scenarios, datasets often contain noisy or mislabeled samples that can significantly impact model performance. Traditional approaches to handling noisy data, such as static pruning methods, often fail to capture the dynamic nature of the learning process and may discard valuable training examples.

Recent work has introduced scoring mechanisms like EL2N (Error L2-Norm) to identify potentially noisy samples. However, these methods typically implement one-time pruning decisions early in training, failing to adapt to the model's evolving learning dynamics. Our work addresses this limitation by introducing a dynamic pruning strategy that adjusts its behavior throughout the training process.

Our main contributions include: a novel dynamic pruning algorithm that adapts its strategy across different training phases; an empirical demonstration of improved robustness to label noise compared to static pruning methods; a comprehensive analysis of pruning dynamics and their impact on model performance

\section{Related Work}

Prior research in dataset selection [4][5] and pruning has evolved across multiple domains and approaches. The foundational work 'Deep Learning on a Data Diet [1]' introduced important scoring mechanisms (GraNd and EL2N) that could identify valuable training examples early in the training process. This work demonstrated that models could maintain or even improve accuracy while training on only half of the CIFAR-10 dataset. However, these static scoring methods showed significant limitations when dealing with noisy data.

Building on these findings, researchers explored dynamic pruning strategies that adapted to the model's training trajectory. This advancement recognized that some training examples (termed "sometimes samples") might be crucial during specific phases of training but less important in others. This dynamic approach proved successful, cutting training time in half for CIFAR datasets while maintaining model performance. The effectiveness of these pruning techniques was further validated beyond computer vision, with successful adaptations for NLP tasks and pre-trained language models like BERT, demonstrating the broad applicability of these principles across different domains.

Our approach is needed because existing methods still face significant challenges when dealing with noisy datasets. While previous work has established the value of data pruning, most approaches use static scoring methods that can't adapt to changing data importance during training. Our dynamic EL2N-based approach directly addresses these limitations.

\section{Methodology}

The core idea of our approach is to dynamically adjust dataset pruning throughout the training process, adapting to both the model's learning progress and the presence of noisy samples. This methodology builds upon the observation that different training examples have varying importance at different stages of the learning process, similar to curriculum learning principles. The algorithm allows retention of informative, low-noise examples that contribute to better parameter updates, while avoiding harmful gradient signals from noisy samples.

The foundation of our approach relies on the Error L2-Norm (EL2N) score, which measures the squared L2 distance between a model's predicted probabilities and the target labels. Formally, for a given example $x$ with label $y$, the EL2N score is defined as:

$EL2N(x,y) = |f_\theta(x) - y|_2$

where $f_\theta(x)$ represents the model's output probabilities. Higher EL2N scores typically indicate either difficult examples or potentially noisy labels, while lower scores suggest examples the model learns more consistently from.

Our method leverages several key heuristics derived from empirical observations in deep learning:

\begin{itemize}
    \item Similar to how curriculum learning exposes the model to easier examples first, early training benefits from cleaner, more representative examples to establish stable feature representations.
    \item Learning dynamics shift during training, early phases focus on capturing broad patterns, while later stages refine performance.
    \item Maintaining a balance between example difficulty and label noise is crucial for optimal learning.
\end{itemize}

The algorithm implementation follows three distinct phases:

\textbf{Phase 1: Initial Noise Reduction (0-30\% of epochs)} 
\begin{itemize} 
    \item Compute EL2N scores for training examples 
    \item Remove top x\% of examples with highest scores (likely noisy samples) 
    \item Upweight remaining examples with low to mid-range scores 
\end{itemize}

\textbf{Phase 2: Mid-Training Transition (30-100\% of epochs)} 
\begin{itemize} 
    \item Gradually increase pruning threshold 
    \item Begin incorporating examples with higher EL2N scores 
    \item Adjust the weights of the retained samples so they are appropriately balanced in the training process
    \item Maintain dynamic balance between example difficulty and noise level 
\end{itemize}

This phased approach allows for robust handling of noisy datasets while maintaining efficient training dynamics and strong generalization performance.

\subsection{Algorithm Details}
The project extends and modifies the code from Paul Mansheej at: \url{https://github.com/mansheej/data_diet}. The code is committed to main at: \url{https://github.com/SidRedPat/data_diet_noisy_260d}.

One of the initial tasks was the implementation of the EL2N score (Error L2-norm), which was not present in the original code:

\begin{lstlisting}[
    language=Python,
    numbers=left,
    numberstyle=\tiny,
    basicstyle=\ttfamily\small,
    breaklines=true,
    frame=single,
    commentstyle=\color{gray},
    keywordstyle=\color{blue},
    stringstyle=\color{green},
    showstringspaces=false
]
def compute_el2n_scores(self, state, f_train, x, y) -> np.ndarray:
    """Compute EL2N scores using JAX/Flax model"""
    with tf.device(self.device):
        # Get model predictions
        logits, _ = f_train(state.params, state.model_state, x)

        # Convert logits to probabilities using JAX's softmax
        probabilities = jax.nn.softmax(logits)

        # Convert targets to one-hot using JAX
        targets_one_hot = y

        # Ensure targets_one_hot has the same shape as probabilities (128, 10)
        if len(targets_one_hot.shape) > 2:
            targets_one_hot = targets_one_hot.reshape(targets_one_hot.shape[0], -1)

        # Compute EL2N score: squared L2 norm of (probabilities - targets)
        errors = probabilities - targets_one_hot
        el2n_scores = jnp.sum(errors**2, axis=-1)

        # Convert to numpy array for consistent processing
        return np.array(el2n_scores)
\end{lstlisting}

Next, we change the training algorithm to adaptively perform pruning during the steps:
\begin{lstlisting}[
    language=Python,
    numbers=left,
    numberstyle=\tiny,
    basicstyle=\ttfamily\small,
    breaklines=true,
    frame=single,
    commentstyle=\color{gray},
    keywordstyle=\color{blue},
    stringstyle=\color{green},
    showstringspaces=false
]
# ...
for t, idxs, x, y in train_batches(I_train, X_train, Y_train, args):
    # Compute EL2N scores and get pruning mask
    el2n_scores = pruning_manager.compute_el2n_scores(state, f_train, x, y)
    mask, weights = pruning_manager.adaptive_pruning_strategy(
        el2n_scores, t, args.log_steps)

    # Apply mask to current batch
    weights = weights[mask]
    x = x[mask]
    y = y[mask]

    # Skip batch if all examples were pruned
    if len(x) == 0:
        continue

    lr = lr_schedule(t)
    loss, acc, logits, new_model_state, gradients = train_step(
        state, x, y, lr, weights)
#...
\end{lstlisting}

To track the different phases of pruning, we make use of a new TrainState variable:

\begin{verbatim}
    state = TrainState(
        step=state.step + 1,
        params=new_params,
        opt_state=new_opt_state,
        model_state=new_model_state,
    )
\end{verbatim}

The core of the algorithm logic is implemented in the \textit{AdaptiveEL2NPruning::adaptive\_pruning\_strategy} method. 

This method introduces several key innovations:

\subsubsection{Early Training Phase (first 30\% of epochs):}
Focuses on noise reduction by removing samples with the highest EL2N scores. It does so by maintaining a fixed pruning percentage (initial\_prune\_percent), preserving samples with lower EL2N scores, which are likely to be clean examples.

\subsubsection{Late Training Phase (remaining 70\% of epochs):}
This phase implements a progressive pruning strategy, dynamically increasing the pruning percentage according to: 

\begin{verbatim}
prune_percent = min(0.5, initial_prune_percent \cdot (1 + 2(epoch_progress - 0.5)))
\end{verbatim}

The objective is to retains samples with higher EL2N scores to focus on harder, more informative examples.

\subsubsection{Sample Weight Compensation:}
Finally, the algorithm implements adaptive weight adjustment for the remaining samples (after pruning is performed). 

This is done by upweighting retained examples, according to: 
\begin{verbatim}
    weights = (
        weights * (total_samples / weights.sum()) if weights.sum() > 0 else weights
    )
\end{verbatim}

This maintains the effective sample size by scaling weights to ensure the learning signal strength remains consistent despite reduced sample count.
    
\section{Experimental Results}

We evaluate our adaptive EL2N-based pruning approach against two baselines: (1) standard training with the full dataset and (2) static EL2N pruning with 25\% data removal. To be able to evaluate our adaptive pruning algorithm, we implemented a noise generation function, which is used at varying levels.

The details are as follows:

\subsection{Setup}
We conducted experiments using:
\begin{itemize}
\item Dataset: CIFAR-10 with artificial label noise (10\%, 15\%, 20\%, 30\%)
\item Architecture: ResNet18
\item Training: 30 epochs, batch size 1024
\item SGD with Nesterov momentum and weight decay
\item learning rate decay starting at 0.1 and following a piecewise schedule by a factor of 0.2
\item Hardware: NVIDIA A100 GPU
\end{itemize}

\subsection{Classification Accuracy}

Our method demonstrates robust performance across different noise levels:

\begin{itemize}
    \item At 20\% noise, adaptive pruning achieves 81\% accuracy, compared to 75\% for full training and 71\% for static EL2N pruning
    \item Performance degradation is more graceful with increasing noise levels compared to static pruning
    \item The accuracy gap between our method and full training remains nearly constant until up to 20\% noise
\end{itemize}

\subsection{Training Dynamics}

The learning curves reveal several key insights:

\begin{itemize}
    \item Early training phase (first 30\% of epochs): 
        \begin{itemize}
            \item Focuses on removing high-noise samples
            \item Maintains stable learning progression
        \end{itemize}
    \item Late training phase:
        \begin{itemize}
            \item Progressive pruning maintains model performance
            \item Avoids the accuracy degradation seen in static pruning
        \end{itemize}
\end{itemize}

\subsection{Pruning Effectiveness}

Our dynamic pruning strategy achieves:

\begin{itemize}
    \item Average pruning rate of 24.4\% across training
    \item Comparable training time to baseline (no significant overhead)
    \item Better accuracy-efficiency trade-off compared to static pruning
\end{itemize}

\subsection{Comparative Analysis}

Table~\ref{tab:results} summarizes the test accuracy across different noise levels \ref{fig:perf-noise}:

\begin{table}[h]
\caption{Test Accuracy (\%) at Different Noise Levels}
\label{tab:results}
\centering
\begin{tabular}{lccccc}
\toprule
Method & 10\% & 15\% & 20\% & 30\% \\
\midrule
Full Training & 85 & 80 & 75 & 67 \\
Static EL2N & 79 & 76 & 71 & 63 \\
Adaptive EL2N & 86 & 84 & 81 & 83 \\
\bottomrule
\end{tabular}
\end{table}

The results demonstrate that our dynamic pruning approach:
\begin{itemize}
    \item Maintains performance close to full training
    \item Significantly outperforms static pruning
    \item Provides consistent benefits across noise levels
    \item Achieves these improvements while training on a reduced dataset
\end{itemize}

These findings support our hypothesis that dynamic adaptation of pruning strategies throughout training can effectively handle noisy datasets while maintaining model performance.

\section{Project Challenges}

Our research encountered several significant challenges that merit discussion. These challenges not only shaped our experimental approach but also highlight important considerations for future work in dynamic dataset pruning.

\subsection{Computational Resources and Time Constraints}

Training deep learning models with various noise levels and pruning strategies required substantial computational resources. Even with access to A100 GPUs, executing more than 12 experimental runs across a two-week period proved time-intensive. Each run needed to be carefully monitored and validated, limiting our ability to explore a broader range of hyperparameters and architectures.

\subsection{Hyperparameter Complexity}

The introduction of dynamic pruning added several new hyperparameters to an already complex training process:

\begin{itemize}
    \item Initial pruning percentage
    \item Transition timing between pruning phases
    \item Rate of pruning adjustment
    \item Interaction with existing hyperparameters (learning rate, batch size)
\end{itemize}

The potential combinations grew exponentially, making exhaustive exploration impractical. The interaction between pruning strategies and traditional training parameters remains an open question.

\subsection{Evaluation}

Noisy datasets complicated the evaluation process in several ways:

\begin{itemize}
    \item Test accuracy exhibited non-monotonic behavior (e.g., better performance at 2K steps than at 4K steps with 20\% noise)
    \item Traditional metrics became less reliable as noise levels increased
    \item Difficulty in distinguishing between pruning effects and noise effects
\end{itemize}

Our primary hypothesis---that dynamic tuning of pruning-related hyperparameters throughout training improves accuracy for noisy datasets---relied heavily on heuristics. This necessitated extensive empirical validation, which was constrained by the aforementioned computational limitations.

These challenges underscore the need for:

\begin{enumerate}
    \item More efficient methods for hyperparameter optimization
    \item Better theoretical understanding of pruning dynamics
    \item Improved metrics for evaluating performance on noisy datasets
    \item Standardized benchmarks for comparing dynamic pruning strategies
\end{enumerate}

Despite these challenges, our results demonstrate the potential of dynamic pruning strategies. Future work will need to address these limitations systematically, potentially through automated hyperparameter optimization and more sophisticated evaluation metrics.

\section{Future Work}

Our current results, while promising, suggest several compelling directions for future research. First, extending the experimental validation would strengthen our findings. This includes conducting additional runs to establish statistical significance and eliminate potential random effects in our current results. More comprehensive validation could also involve scaling to larger datasets such as CIFAR-100, which would help verify the approach's effectiveness on more complex classification tasks.

A particularly interesting direction involves extending our dynamic pruning framework beyond vision tasks. Specifically, adapting the method for different architectures like GPT and exploring its effectiveness on text, voice, and video data would demonstrate the framework's generalizability across domains. This cross-domain validation is especially important given the prevalence of noisy labels in real-world datasets across various modalities.

From an algorithmic perspective, several optimizations merit investigation. The current pruning percentage settings across different training phases, while effective, could benefit from more principled tuning approaches. We envision developing methods for automatic hyperparameter optimization specifically focused on the newly introduced parameters in our framework.

While we have focused primarily on improving training efficiency in the presence of noisy data, there are potentially important implications for bias reduction in machine learning models. Specifically, investigating how dynamic pruning affects representation of minority groups in the training data could yield insights into creating more equitable learning systems.

\section{Conclusion}
We presented a dynamic EL2N-based pruning approach that effectively handles noisy datasets while maintaining model performance. Our method demonstrates significant improvements over static pruning approaches, particularly in high-noise scenarios. The results suggest that adaptive pruning strategies can better capture the changing dynamics of model training, leading to more robust and efficient learning.

\section*{References}
{
\small

[1] Paul et al. "Deep Learning on a Data Diet: Finding Important Examples Early in Training." arXiv:2107.07075, 2021.

[2] Kirsch, A. "Does 'Deep Learning on a Data Diet' reproduce?" OATML Technical Report, 2023.

[3] Cody Coleman, Christopher Yeh, Stephen Mussmann, Baharan Mirzasoleiman, Peter Bailis, Percy Liang, Jure Leskovec, Matei Zaharia. "Selection via Proxy: Efficient Data Selection for Deep Learning". arXiv:1906.11829, 2019.

[4] Baharan Mirzasoleiman, Jeff Bilmes, Jure Leskovec. "Coresets for Data-efficient Training of Machine Learning Models". arXiv:1906.01827, 2019.

[5] Baharan Mirzasoleiman, Kaidi Cao, Jure Leskovec. "Coresets for Robust Training of Neural Networks against Noisy Labels". arXiv:2011.07451, 2020.

[6] Bertsekas, D. P. "Incremental gradient, subgradient, and proximal methods for convex optimization: A survey". arXiv preprint arXiv:1507.01030, 2015b.

[7] Glorot, X. and Bengio, Y. "Understanding the difficulty of training deep feedforward neural networks." In Proceed- ings of the thirteenth international conference on artifi- cial intelligence and statistics, pp. 249–256, 2010.

[8] Kaufman, L., Rousseeuw, P., and Dodge, Y. "Clustering by means of medoids in statistical data analysis", 1987.

[9] Lin, H., Bilmes, J., and Xie, S. "Graph-based submodular selection for extractive summarization". In Proc. IEEE Au- tomatic Speech Recognition and Understanding (ASRU), Merano, Italy, December 2009.

}

\appendix
\section{Additional Material}

\begin{figure}
    \includegraphics[scale=0.3]{graph-perf-noise.png}
    \caption{Performance vs Noise Level}
    \label{fig:perf-noise}
\end{figure}

\end{document}
